\documentclass{uwstat572}

%%\setlength{\oddsidemargin}{0.25in}
%%\setlength{\textwidth}{6in}
%%\setlength{\topmargin}{0.5in}
%%\setlength{\textheight}{9in}

\renewcommand{\baselinestretch}{1.5} 
\usepackage{/Users/paigejo/mystyle}

\bibliographystyle{plainnat}

\usepackage{color}
\usepackage{ulem}
\usepackage{subfig}
\usepackage{lscape}
\usepackage{rotating}
\usepackage{placeins}

\newcommand{\diag}[1]{\text{diag}\paren{#1}}

\begin{document}

\begin{center}
  {\LARGE Title}\\\ \\
  {John Paige \\ 
    Department of Statistics, University of Washington Seattle, WA, 98195, USA
  }
\end{center}

\abstract{
Abstract
}

\section{Model}
\label{model}
For the strain rates estimated from the GPS data:
$X_i = R_i + e_i$\\
for:\\
$X_i$: $i$th observed strain rate from GPS data\\
$\vec{R} \sim LN(\log(\vec{\eta}) - \diag{\Sigma_R}/2, \Sigma_R)$: true state of present strain rate where the average strain rate is $\vec{\eta}$.  Note that this is an exponentiated GP.\\
$\vec{e} \sim GP(\vec{0}, \rho(\cdot))$: measurement error with Matern correlation fit from the $\vec{X}$ observations.  The estimated errors of the observations, $\vec{X}$, could be used for a simple nonstationary variance model to scale the correlation matrix by the standard errors at each location.
\\\\
For the subsidence data:\\
$S_i = \Delta T \cdot \eta_i \xi_i$\\
$Y_i = g_i(\vec{S}) + \varepsilon_i$\\
for:\\
$\vec{S}$: earthquake slip values throughout the fault\\
$\Delta T$: time since previous earthquake\\
$\vec{\xi} \sim LN(-\diag{\Sigma_\xi}/2, \Sigma_\xi)$: multiplicative deviation with unit expectation of slip during earthquake from slip accumulated since last earthquake.  Multiplicative LN variable ensures positive slips.  Note that this is the exponential of a GP.\\
$\vec{Y}$: observed subsidence levels\\
$g_i(\vec{s})$: $i$th subsidence level given earthquake slips, $\vec{s}$.  Compute using Okada model\\
$\vec{\varepsilon} \sim N(\vec{0}, \diag{\vec{\sigma}})$: independent measurement error of subsidence levels where $\vec{\sigma}$ is known.
\\\\
The parameters being estimated are:
$$ \v{\theta} = ($$
From there it seems like it's possible to calculate likelihood of the strain rate data, $\vec{X}$, and the expected likelihood of the subsidence data, $\vec{Y}$, using Monte Carlo sampling.  I can't think of another way to get the likelihood of $\vec{Y}$ aside from getting the expected likelihood using Monte Carlo sampling, since the $g(\cdot)$ function has an unknown affect on the distribution of $\vec{S}$.

%\bibliography{M9}

\end{document}








